\documentclass{article}
\usepackage[left=1in,right=1in,top=1in,bottom=1in]{geometry}
\usepackage{amsmath, amsthm, verbatim, enumerate, xcolor, mathpartir}

\newtheorem{task}{Task}
\newtheorem{lemma}{Lemma}
\newtheorem{thm}{Theorem}

\newcommand{\kw}[1]{\ensuremath{\mathsf{#1}}}

\newcommand{\blank}{\underline{\hspace{0.2cm}\color{red}{?}\hspace{0.2cm}}}

%% Inference rules
%Inference rule with a title
\newcommand{\Rule}[4][]{\ensuremath{\inferrule*[right={(#2)},#1]{#3}{#4}}}
%Inference rule with no title
\newcommand{\RuleNoLabel}[3][]{\ensuremath{\inferrule*[#1]{#2}{#3}}}
\newcommand{\Rulename}[1]{\textsc{#1}}
%% BNF
\newcommand{\bnfdef}{\mathop{::=}}
\newcommand{\bnfalt}{\mathbin{|}}

%% E syntax
\newcommand{\ezlang}{\textsf{EZ}}
\newcommand{\elang}{\textsf{E}}
\newcommand{\cat}{\ensuremath{\mathbin{\text{\textasciicircum}}}}
\newcommand{\kwlen}[1]{\ensuremath{|#1|}}
\newcommand{\kwlet}[3]{\ensuremath{\kw{let}~#1=#2~\kw{in}~#3}}
\newcommand{\kwn}[1]{\ensuremath{\overline{#1}}}
\newcommand{\kws}[1]{\text{``}\ensuremath{#1}\text{''}}
\newcommand{\kwif}[3]{\ensuremath{\kw{if}~#1~\kw{then}~#2~\kw{else}~#3}}
\newcommand{\kwtrue}{\kw{true}}
\newcommand{\kwfalse}{\kw{false}}

\newcommand{\kwint}{\kw{int}}
\newcommand{\kwnz}{\kw{nz}}
\newcommand{\kwmz}{\kw{mz}}

%% STLC/System F
\newcommand{\kwunitt}{\kw{unit}}
\newcommand{\kwarr}[2]{#1 \rightarrow #2}
\newcommand{\kwprod}[2]{#1 \times #2}
\newcommand{\kwsum}[2]{#1 + #2}
\newcommand{\kwvoid}{\kw{void}}
\newcommand{\kwunit}{\kw{()}}
\newcommand{\kwfun}[3]{\lambda #1 : #2. #3}
\newcommand{\kwapp}[2]{#1~#2}
\newcommand{\kwpair}[2]{(#1, #2)}
\newcommand{\kwfst}[1]{\kw{fst}~#1}
\newcommand{\kwsnd}[1]{\kw{snd}~#1}
\newcommand{\kwinl}[1]{\kw{inl}~#1}
\newcommand{\kwinr}[1]{\kw{inr}~#1}
\newcommand{\kwcase}[5]{\kw{case}~#1~\kw{of}~\{#2. #3; #4. #5\}}
\newcommand{\kwabort}[1]{\kw{abort}~#1}
\newcommand{\kwforall}[2]{\ensuremath{\forall #1. #2}}
\newcommand{\kwtfun}[2]{\ensuremath{\Lambda #1. #2}}
\newcommand{\kwtapp}[2]{\ensuremath{#1[#2]}}
  
%% Logic
\newcommand{\limp}{\ensuremath{\Rightarrow}}
\newcommand{\lbimp}{\ensuremath{\Leftrightarrow}}

%% Judgments
\newcommand{\ctx}{\ensuremath{\Gamma}}
\newcommand{\val}{\ensuremath{~\mathsf{val}}}
\newcommand{\step}{\ensuremath{\mapsto}}
\newcommand{\typed}[3]{\ensuremath{#1 \vdash #2 : #3}}
\newcommand{\etyped}[2]{\ensuremath{#1 : #2}}
\newcommand{\sub}[3]{\ensuremath{[#1/#2]#3}}
\newcommand{\fv}[1]{\ensuremath{\mathsf{FV}(#1)}}
\newcommand{\tctx}{\ensuremath{\Delta}}
\newcommand{\ftyped}[4]{\ensuremath{#1; #2 \vdash #3 : #4}}


\title{IIT CS595: Topics \& Applications in Programming Languages\\
  {\large Homework 0: Rule induction, expressions, type safety}}
\author{Dustin Van Tate Testa}

\begin{document}
\maketitle

\section{Logistics}

\paragraph{Writing up Solutions.}
If you're looking at this file, you're probably (at least thinking about)
writing up your solutions in LaTeX. Nice! If you need help, see the writeup
PDF for a link to resources and other options.

\paragraph{Collaboration.}
Please see the collaboration policy on the course website.

\paragraph{Submission.}
Submit your answers as a single .pdf or .doc file (see above) on Blackboard
under
the correct assignment. 


\section{Evaluation}
\begin{task}
  Evaluate the following expression using the dynamics semantics from class,
  showing each step:

  (Put your answers in the ... and copy and paste the line to add more lines
  as necessary).
  
  \[
  \begin{array}{l l}
    &
    \kwlet{x}{\kwn{2}}{x+(\kwlet{x}{\kwn{1}}{\kwlet{y}{x+\kwn{1}}{x+y}})}\\
    \step & \kwn{2}+(\kwlet{x}{\kwn{1}}{\kwlet{y}{x+\kwn{1}}{x+y}})\\
    \step & \kwn{2}+(\kwlet{y}{\kwn{1}+\kwn{1}}{\kwn{1}+y}\\
    \step & \kwn{2}+(\kwlet{y}{\kwn{2}}{\kwn{1}+y}\\
    \step & \kwn{2}+(\kwn{1}+\kwn{2})\\
    \step & \kwn{2}+\kwn{3}\\
    \step & \kwn{5}\\
  \end{array}
  \]
\end{task}

\section{Red-Black Trees}
In data structures, a red-black tree is a binary tree whose nodes are colored
either red or black.
%
Red-black trees must obey the following rules:
\begin{itemize}
\item Leaf nodes must be black.
\item No red node has red children.
\item Any path from the root to a leaf contains the same number of black nodes.
\end{itemize}

We can express red-black trees using the following small language
(note that this syntax doesn't enforce the rules above; we'll do that later).

\[
\begin{array}{r l l l}
  Trees & \tree & \bnfdef & \leaf \bnfalt \bnode{\tree}{\tree} \bnfalt
  \rnode{\tree}{\tree}\\
  Colors & \colr & \bnfdef & \bc \bnfalt \rc\\
\end{array}
\]
where~$\leaf$ is a leaf and~$\bnode{\tree_1}{\tree_2}$
and~$\rnode{\tree_1}{\tree_2}$ represent black and red nodes, respectively,
with children~$\tree_1$ and~$\tree_2$.

We will define a judgment~$\tree\istree{\colr}{n}$, meaning that~$\tree$ is
a valid red-black tree whose root node has the color~$\colr$ (which is
either~$\bc$ or~$\rc$) and all of whose paths from root to leaf have
exactly~$n$ black nodes.

{
  \centering
  \def \MathparLineskip {\lineskip=0.43cm}
  \begin{mathpar}

    \Rule{RBT-1}
         {\strut}
         {\leaf \istree{\bc}{1}}
    \and
    \Rule{RBT-2}
         {\tree_1 \istree{\colr_1}{n}\\
           \tree_2 \istree{\colr_2}{n}}
         {\bnode{\tree_1}{\tree_2} \istree{\bc}{n+1}}
    \and
    \Rule{RBT-3}
         {\tree_1 \istree{\bc}{n}\\
           \tree_2 \istree{\bc}{n}}
         {\rnode{\tree_1}{\tree_2} \istree{\rc}{n}}
  \end{mathpar}
}

Rule~(RBT-1) says that leaves are valid trees as long as they are
black: a leaf by definition has~1 black node on any path from root to leaf.
%
Rule~(RBT-2) allows a black node with two children as long as both
children are valid red-black trees with the same number of black nodes on any
path (regardless of what color their roots are); the result is a tree with a
black root and~$n+1$ black nodes on any path.
%
Rule~(RBT-3) enforces that red nodes do not have red children by
requiring in the premises that both children have black roots.

Define~$\nodes{\tree}$ to be the number of nodes, including leaves, in a
tree:

\[
\begin{array}{l l l}
  \nodes{\leaf} & = & 1\\
  \nodes{\bnode{\tree_1}{\tree_2}} & = & 1 + \nodes{\tree_1} + \nodes{\tree_2}\\
  \nodes{\rnode{\tree_1}{\tree_2}} & = & 1 + \nodes{\tree_1} + \nodes{\tree_2}
\end{array}
\]

\begin{task}
  Prove by rule induction that for any tree~$\tree$,
  color~$\colr \in \{\rc, \bc\}$ and~$n \geq 1$,
  if~$\tree \istree{\colr}{n}$, then~$\nodes{\tree} \geq 2^{n-1}$.

  {\em You must} use rule induction, not any other proof technique.

  \textbf{Note:} This, together with one or two other properties of
  red-black trees, proves that a valid red-black tree is approximately
  balanced.
\end{task}

\textbf{Answer:}

- Notice use of shorthand $\kwlen{\tree}$ to indicate the value of $n$ (black height) for tree \tree \\

Base Case: 
\begin{itemize}
    \item all cases when n = 1:
        \begin{itemize}
            \item {\tree = \leaf}: Rule~(RBT-1) confirms that when T is a single black node, n = 1.\\
            The proposition holds for this case as $\nodes{\leaf} = 1 \geq 2^{n-1} = 2^{1-1} = 1$.

            \item {\tree = \rnode{\leaf}{\leaf}}: rule~(RBT-3) confirms that when $\tree$ is a red node,
            n is the same as it's children, which when they're both leaves makes n = 1.\\
            The proposition holds for this case as follows:\\
            LHS: $\nodes{\rnode{\leaf}{\leaf}} = 1 + \nodes{\leaf} + \nodes{\leaf} = 1 + 1 + 1 = 3$\\
            RHS: $2^{n-1} = 2 ^ {1 - 1} = 2^0 = 1$\\
            giving $3 \geq 1$ 
        \end{itemize}
\end{itemize}

Inductive case assuming $\nodes{\tree} \geq 2^{n-1}$:
\begin{itemize}
    \item Case when $\tree = \bnode{...}{...}$ -- tree's top element is a black node
        \[
        \begin{array}{l l l l}
            \nodes{\bnode{\tree_1}{\tree_2}} & \geq & 2^{\kwlen{\tree}-1} & \text{the proposition for this case}\\
            1 + \nodes{\tree_1} + \nodes{\tree_2} & \geq & 1 + 2 ^ {\kwlen{\tree_1} - 1} + 2 ^ {\kwlen{\tree_2} - 1} & \text{apply rule (RBT-2)}\\
            1 & \geq & 1  & \text{apply inductive hypothesis}\\
        \end{array}
        \]
    
    \item Case when $\tree = \rnode{...}{...}$ -- tree's top element is a red node
        \[
        \begin{array}{l l l l}
            \nodes{\rnode{\tree_1}{\tree_2}} & \geq & 2^{\kwlen{\tree}-1} & \text{the proposition for this case}\\
            1 + \nodes{\tree_1} + \nodes{\tree_2} & \geq & 2 ^ {\kwlen{\tree_1} - 1} + 2 ^ {\kwlen{\tree_2} - 1} & \text{apply rule (RBT-3)}\\
            1 & \geq & 0  & \text{apply inductive hypothesis}\\
        \end{array}
        \]
\end{itemize}

\section{Booleans}

In this task, you'll extend the {\Elang} language from class with Booleans:

\[
\begin{array}{r l l l}
  \mathit{Types} & \tau & \bnfdef &
  \kwint \bnfalt \kwstring \bnfalt \kwbool\\
  \mathit{Expressions} & e & \bnfdef &
  x \bnfalt
  \kwn{n} \bnfalt
  \kws{s} \bnfalt
  \kwtrue \bnfalt
  \kwfalse \bnfalt
  e = e \bnfalt
  e + e \bnfalt
  e \cat e \bnfalt
  \kwlen{e} \bnfalt
  \kwlet{x}{e}{e} \bnfalt
  \kwif{e}{e}{e}
\end{array}
\]

The expressions~$\kwtrue$ and~$\kwfalse$ are values and have their usual
meanings.
%
The expression~$\kwif{e_1}{e_2}{e_3}$ should evaluate~$e_1$. If it evaluates
to~$\kwtrue$, it should continue evaluating~$e_2$, otherwise it should
continue evaluating~$e_3$.
%
The other branch {\em should not} be evaluated (this isn't possible to express
in {\Elang}, but consider the code $\kwif{x = \kwn{0}}{\kwn{0}}{\kwn{42} / x}$: we certainly
don't want to evaluate the else branch when the condition is true!).
%
We've also added an integer equality test~$e_1 = e_2$ to have an interesting
way of producing Booleans (this is an operation on integers only, not on
Booleans or strings).
%
Recall the dynamic semantics from class, now extended with the rules
for~$e_1 = e_2$.

\textbf{Note:} There's a lot here, but don't panic:
the only new rules here are (S-11) through (S-14). The rest are just there
as a reminder. Your job is only to add Booleans to the language.

{
  \centering
  \def \MathparLineskip {\lineskip=0.43cm}
  \begin{mathpar}
    \Rule{V-1}
         {\strut} % This leaves the proper spacing above the line of an axiom
         {\kwn{n}\val}
    \and
    \Rule{V-2}
         {\strut}
         {\kws{s}\val}
    \and
    \Rule{S-1}
         {\strut}
         {\kwn{n_1} + \kwn{n_2} \step \kwn{n_1 + n_2}}
    \and
    \Rule{S-2}
         {\strut}
         {\kws{s_1} \cat \kws{s_2} \step \kws{s_1s_2}}
    \and
    \Rule{S-3}
         {\strut}
         {\kwlen{\kws{s}} \step \kwn{\kwlen{s}}}
    \and
    \Rule{S-4}
         {e_1 \step e_1'}
         {e_1 + e_2 \step e_1' + e_2}
    \and
    \Rule{S-5}
         {e_2 \step e_2'}
         {\kwn{n_1} + e_2 \step \kwn{n_1} + e_2'}
    \and
    \Rule{S-6}
         {e_1 \step e_1'}
         {e_1 \cat e_2 \step e_1' \cat e_2}
    \and
    \Rule{S-7}
         {e_2 \step e_2'}
         {\kws{s_1} \cat e_2 \step \kws{s_1} \cat e_2'} 
    \and
    \Rule{S-8}
         {e \step e'}
         {\kwlen{e} \step \kwlen{e'}}
    \and
    \Rule{S-9}
         {e_1 \step e_1'}
         {\kwlet{x}{e_1}{e_2} \step \kwlet{x}{e_1'}{e_2}}
    \and
    \Rule{S-10}
         {v \val}
         {\kwlet{x}{v}{e_2} \step \sub{v}{x}{e_2}}
    \and
    \Rule{S-11}
         {e_1 \step e_1'}
         {e_1 = e_2 \step e_1' = e_2}
    \and
    \Rule{S-12}
         {e_2 \step e_2'}
         {\kwn{n_1} = e_2 \step \kwn{n_1} = e_2'}
    \and
    \Rule{S-13}
         {\strut}
         {\kwn{n} = \kwn{n} \step \kwtrue}
    \and
    \Rule{S-14}
         {n_1 \neq n_2}
         {\kwn{n_1} = \kwn{n_2} \step \kwfalse}
  \end{mathpar}
}

\begin{task}\label{task:dyn}
  Write the new inference rules for the dynamic semantics of Booleans and the
  if-else construct.
  You should have 2 new rules for the judgment~$e\val$ and 3 new rules
  for the judgment~$e \step e$.\\
  (\textbf{Hint:} only one of these will be a search rule.)
\end{task}

\textbf{Answer:}

{
    \centering
    \def \MathparLineskip{\lineskip=0.43cm}
    \begin{mathpar}
        \Rule{V-3}
            {\strut}
            {\kwfalse \val}
        \and
        \Rule{V-4}
            {\strut}
            {\kwtrue \val}
        \and
        \Rule{S-15}
            {e_1 \step e_1'}
            {\kwif{e_0}{e_1}{e_2} \step \kwif{e_0'}{e_1}{e_2}}
        \and
        \Rule{S-16}
            {\strut}
            {\kwif{\kwtrue}{e_1}{e_2} \step e_1}
        \and
        \Rule{S-17}
            {\strut}
            {\kwif{\kwfalse}{e_1}{e_2} \step e_2}
    \end{mathpar}
}

We also extend the typing rules with the new rule for equality testing and
Booleans.

{
  \centering
  \def \MathparLineskip {\lineskip=0.43cm}
  \begin{mathpar}
    \Rule{T-8}
         {\typed{\ctx}{e_1}{\kwint}\\
           % Use \\ to separate multiple premises
           \typed{\ctx}{e_2}{\kwint}}
         {\typed{\ctx}{e_1 = e_2}{\kwbool}}
    \and
    \Rule{T-9}
         {\strut}
         {\typed{\ctx}{\kwtrue}{\kwbool}}
    \and
    \Rule{T-10}
         {\strut}
         {\typed{\ctx}{\kwfalse}{\kwbool}}
    \and
    \Rule{T-11}
         {\typed{\ctx}{e_1}{\kwbool}\\
           \typed{\ctx}{e_2}{\tau}\\
           \typed{\ctx}{e_3}{\tau}}
         {\typed{\ctx}{\kwif{e_1}{e_2}{e_3}}{\tau}}
  \end{mathpar}
}

Note that rule (T-11) doesn't require that the two branches of the conditional
have a particular type (e.g.~$\kwint$).
%
The use of the same metavariable~$\tau$ {\em does} however, mean that the
two branches~$e_2$ and~$e_3$ must have the {\em same} type, which is then the
type of the whole expression
(this makes sense: how could we possibly give a type to the expression
$\kwif{n = \kwn{0}}{\kwn{42}}{\kws{\text{Oops}}}$?).

\begin{task}
  Prove the cases of the Preservation theorem for the new rules you
  added in Task~\ref{task:dyn}.
\end{task}

\textbf{Answer:}

Proposition: if $\typed{\ctx}{e}{\tau}$ and $e \step e'$ then $\typed{\ctx}{e'}{\tau}$
\begin{itemize}
    \item S-16: When $e = \kwif{\kwtrue}{e_1}{e_2}$\\ \step $e' = e_1$\\
        by inversion on rule (T-11) $\typed{\ctx}{\kwif{e_0}{e_1}{e_2}}{\tau}$ 
            and  $\typed{\ctx}{e_1}{\tau}$\\
        with induction on $e_1 \step e_1'$ this confirms that the type of the stepped value is preserved.
        
    \item S-17: When $e = \kwif{\kwfalse}{e_1}{e_2}$\\ \step $e' = e_2$\\
        by inversion on rule (T-11) $\typed{\ctx}{\kwif{e_0}{e_1}{e_2}}{\tau}$ 
            and  $\typed{\ctx}{e_2}{\tau}$\\
        with induction on $e_2 \step e_2'$ this confirms that the type of the stepped value is preserved.
    
    \item S-15: When $e = \kwif{e_0}{e_1}{e_2}$\\ \step $e' = \kwif{e_0'}{e_1}{e_2}$\\
        by inversion on rule (T-11), $\typed{\ctx}{e_0}{\kwbool}$, $\typed{\ctx}{e_1}{\tau}$, 
            and $\typed{\ctx}{e_2}{\tau}$\\
        by induction we can assume $e_0 \step e_0'$ with $\typed{\ctx}{e_0'}{\kwbool}$\\
        thus we can apply (T-11) to assert that the type is the same.
\end{itemize}

\begin{task}\label{task-cf}
  State (you do not have to prove it) the new case of the Canonical Forms
  lemma for Booleans.
\end{task}

\textbf{Answer:}
if $e \val$ and $\typed{}{e}{\kwbool}$ then $e$ is either $\kwtrue$ or $\kwfalse$.


\begin{task}
  Prove the cases of the Progress theorem for the new rules
  (T-9) through (T-11). You may (and should) use the new case of Canonical
  Forms from Task~\ref{task-cf}.
\end{task}

\textbf{Answer:}

Proposition: if $\typed{}{e}{\tau}$ then $e \val$ or $e \step e'$
\begin{itemize}
    \item T-9: $e = \kwtrue$, $\tau = \kwbool$\\
        by V-4 $e \val$
    \item T-10: $e = \kwfalse$, $\tau = \kwbool$\\
        by V-3 $e \val$
    \item T-11: $e = \kwif{e_1}{e_2}{e_3}$\\
        by inversion on rule T-11 $\typed{}{e_1}{\kwbool}$\\
        by induction, either $e_1 \val$ or $e_1 \step e_1'$\\
        by the Canonical Forms lemma for from Task~\ref{task-cf} $e_1$ can hold either \kwtrue or \kwfalse
        \begin{itemize}
            \item \kwtrue : S-16 confirms that $\kwif{\kwtrue}{e_2}{e_3} \step e_2$ and thus $e' = e_2$
            \item \kwfalse : S-17 confirms that $\kwif{\kwfalse}{e_2}{e_3} \step e_3$ and thus $e' = e_3$
        \end{itemize}

\end{itemize}

\end{document}
